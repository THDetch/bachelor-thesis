\chapter{Intermediate Outputs during Training}
\label{appendix:intermediate_visuals}
\begin{figure}[!htbp]
  \centering
  \renewcommand{\arraystretch}{1.0}
  \adjustbox{max width=\textwidth, max totalheight=0.7\textheight, keepaspectratio}{
    \begin{tabular}{c c@{\hspace{40pt}}c}
      \rotatebox{90}{\hspace{3pt}\textbf{SAR RGB}} &
      \includegraphics[width=0.45\linewidth]{img/intermediate_gt_imgs/sample_05_sarRGB.png} &
      \includegraphics[width=0.45\linewidth]{img/intermediate_gt_imgs/sample_04_sarRGB.png} \\
      \rotatebox{90}{\hspace{3pt}\textbf{Epoch 1}} &
      \includegraphics[width=0.45\linewidth]{img/intermediate_gt_imgs/001_sample_05_pred.png} &
      \includegraphics[width=0.45\linewidth]{img/intermediate_gt_imgs/001_sample_04_pred.png} \\
      \rotatebox{90}{\hspace{3pt}\textbf{Epoch 51}} &
      \includegraphics[width=0.45\linewidth]{img/intermediate_gt_imgs/051_sample_05_pred.png} &
      \includegraphics[width=0.45\linewidth]{img/intermediate_gt_imgs/051_sample_04_pred.png} \\
      \rotatebox{90}{\hspace{3pt}\textbf{Epoch 101}} &
      \includegraphics[width=0.45\linewidth]{img/intermediate_gt_imgs/101_sample_05_pred.png} &
      \includegraphics[width=0.45\linewidth]{img/intermediate_gt_imgs/101_sample_04_pred.png} \\
      \rotatebox{90}{\hspace{3pt}\textbf{Epoch 150}} &
      \includegraphics[width=0.45\linewidth]{img/intermediate_gt_imgs/150_sample_05_pred.png} &
      \includegraphics[width=0.45\linewidth]{img/intermediate_gt_imgs/150_sample_04_pred.png} \\
      \rotatebox{90}{\hspace{3pt}\textbf{Ground truth}} &
      \includegraphics[width=0.45\linewidth]{img/intermediate_gt_imgs/sample_05_tgt.png} &
      \includegraphics[width=0.45\linewidth]{img/intermediate_gt_imgs/sample_04_tgt.png} \\
    \end{tabular}
  } % end adjustbox
  \caption[Intermediate results during training]{Evolution of generated optical images across training epochs for two samples (columns). Rows show pseudo-RGB SAR input (R: VV, G: VH, B: VV7VH), generated optical images after epochs 1, 51, 101, 150, and the ground truth.}
  \label{fig:intermediate_training}
\end{figure}

\chapter{Bandwise Grayscale Reconstructions}
\label{appendix:bandwise_results}

\begin{figure}[h!]
    \centering
    \captionsetup[subfigure]{labelformat=empty}

    % ---------- Row 1 ----------
    \begin{subfigure}{0.48\textwidth}
        \centering
        \includegraphics[width=\linewidth]{img/bands_gray/sample_000008_B01_panel.png}
        \caption{B1 (Aerosols, 443 nm)}
    \end{subfigure}\hfill
    \begin{subfigure}{0.48\textwidth}
        \centering
        \includegraphics[width=\linewidth]{img/bands_gray/sample_000008_B02_panel.png}
        \caption{B2 (Blue, 490 nm)}
    \end{subfigure}

    % ---------- Row 2 ----------
    \vspace{0.5em}
    \begin{subfigure}{0.48\textwidth}
        \centering
        \includegraphics[width=\linewidth]{img/bands_gray/sample_000008_B03_panel.png}
        \caption{B3 (Green, 560 nm)}
    \end{subfigure}\hfill
    \begin{subfigure}{0.48\textwidth}
        \centering
        \includegraphics[width=\linewidth]{img/bands_gray/sample_000008_B04_panel.png}
        \caption{B4 (Red, 665 nm)}
    \end{subfigure}

    % ---------- Row 3 ----------
    \vspace{0.5em}
    \begin{subfigure}{0.48\textwidth}
        \centering
        \includegraphics[width=\linewidth]{img/bands_gray/sample_000008_B05_panel.png}
        \caption{B5 (Red Edge, 705 nm)}
    \end{subfigure}\hfill
    \begin{subfigure}{0.48\textwidth}
        \centering
        \includegraphics[width=\linewidth]{img/bands_gray/sample_000008_B06_panel.png}
        \caption{B6 (Red Edge, 740 nm)}
    \end{subfigure}

    \caption[Bandwise grayscale reconstructions (Bands 1–6)]%
    {Generated (left) and ground-truth (right) grayscale representations for Sentinel-2 Bands~1–6.}
    \label{fig:appendix_band_panels}
\end{figure}

\begin{figure}[p]
    \ContinuedFloat
    \centering
    \captionsetup[subfigure]{labelformat=empty}

    % ---------- Row 4 ----------
    \begin{subfigure}{0.48\textwidth}
        \centering
        \includegraphics[width=\linewidth]{img/bands_gray/sample_000008_B07_panel.png}
        \caption{B7 (Red Edge, 783 nm)}
    \end{subfigure}\hfill
    \begin{subfigure}{0.48\textwidth}
        \centering
        \includegraphics[width=\linewidth]{img/bands_gray/sample_000008_B08_panel.png}
        \caption{B8 (NIR, 842 nm)}
    \end{subfigure}

    % ---------- Row 5 ----------
    \vspace{0.5em}
    \begin{subfigure}{0.48\textwidth}
        \centering
        \includegraphics[width=\linewidth]{img/bands_gray/sample_000008_B09_panel.png}
        \caption{B8A (Red Edge, 865 nm)}
    \end{subfigure}\hfill
    \begin{subfigure}{0.48\textwidth}
        \centering
        \includegraphics[width=\linewidth]{img/bands_gray/sample_000008_B10_panel.png}
        \caption{B9 (Water Vapour, 945 nm)}
    \end{subfigure}

    % ---------- Row 6 ----------
    \vspace{0.5em}
    \begin{subfigure}{0.48\textwidth}
        \centering
        \includegraphics[width=\linewidth]{img/bands_gray/sample_000008_B11_panel.png}
        \caption{B10 (Cirrus, 1375 nm)}
    \end{subfigure}\hfill
    \begin{subfigure}{0.48\textwidth}
        \centering
        \includegraphics[width=\linewidth]{img/bands_gray/sample_000008_B12_panel.png}
        \caption{B11 (SWIR, 1610 nm)}
    \end{subfigure}

    % ---------- Row 7 ----------
    \vspace{0.5em}
    \begin{subfigure}{0.48\textwidth}
        \centering
        \includegraphics[width=\linewidth]{img/bands_gray/sample_000008_B13_panel.png}
        \caption{B12 (SWIR, 2190 nm)}
    \end{subfigure}

    \caption[Bandwise grayscale reconstructions (Bands 7–12)]%
    {Generated (left) and ground-truth (right) grayscale representations for Sentinel-2 Bands~7–12.}
\end{figure}

\chapter{Results Across Different Seasons}
\label{appendix:results_seasons}

% ---------------- FALL ----------------
\begin{table}[h!]
\centering
\caption[Quantitative results on the fall subset]{Quantitative performance of the model on the fall subset of the SEN12-MS dataset.}
\begin{tabular}{lcccc}
\toprule
\textbf{Season} & \textbf{SSIM} & \textbf{PSNR (dB)} & \textbf{LPIPS} & \textbf{SAM (°)} \\
\midrule
Fall & 0.788 & 22.40 & 0.269 & 13.67 \\
\bottomrule
\end{tabular}
\label{tab:fall_results}
\end{table}

% \vspace{1em} 

\begin{figure}[h!]
    \centering
    \setlength{\tabcolsep}{2pt}
    \renewcommand{\arraystretch}{1.0}
    \begin{tabular}{c *{3}{c}}
        
        \includegraphics[width=0.25\textwidth]{img/seasons/fall/sample_000031_sar_pseudo.png} &
        \includegraphics[width=0.25\textwidth]{img/seasons/fall/sample_000031_pred_rgb.png} &
        \includegraphics[width=0.25\textwidth]{img/seasons/fall/sample_000031_true_rgb.png} \\
        
        \includegraphics[width=0.25\textwidth]{img/seasons/fall/sample_000019_sar_pseudo.png} &
        \includegraphics[width=0.25\textwidth]{img/seasons/fall/sample_000019_pred_rgb.png} &
        \includegraphics[width=0.25\textwidth]{img/seasons/fall/sample_000019_true_rgb.png} \\
        
        \includegraphics[width=0.25\textwidth]{img/seasons/fall/sample_000015_sar_pseudo.png} &
        \includegraphics[width=0.25\textwidth]{img/seasons/fall/sample_000015_pred_rgb.png} &
        \includegraphics[width=0.25\textwidth]{img/seasons/fall/sample_000015_true_rgb.png} \\

    \end{tabular}
    \caption[Qualitative results on the fall subset]{%
    Representative SAR-to-optical translation result on the \textbf{fall} subset from the SEN12-MS dataset. 
    Columns: \textbf{(i)} SAR input (pseudo-RGB), 
    \textbf{(ii)} model-generated optical image, 
    \textbf{(iii)} ground-truth Sentinel-2 image.
    }
    \label{fig:appendix_fall}
\end{figure}

\newpage
% ---------------- SPRING ----------------
% \section{Spring Subset}
\vspace{2em} 

\begin{table}[h!]
\centering
\caption[Quantitative results on the spring subset]{Quantitative performance of the model on the spring subset of the SEN12-MS dataset.}
\begin{tabular}{lcccc}
\toprule
\textbf{Season} & \textbf{SSIM} & \textbf{PSNR (dB)} & \textbf{LPIPS} & \textbf{SAM (°)} \\
\midrule
Spring & 0.791 & 20.00 & 0.297 & 12.06 \\
\bottomrule
\end{tabular}
\label{tab:spring_results}
\end{table}

\vspace{2em} 

\begin{figure}[h!]
    \centering
    \setlength{\tabcolsep}{2pt}
    \renewcommand{\arraystretch}{1.0}
    \begin{tabular}{c *{3}{c}}
        \includegraphics[width=0.25\textwidth]{img/seasons/spring/sample_000011_sar_pseudo.png} &
        \includegraphics[width=0.25\textwidth]{img/seasons/spring/sample_000011_pred_rgb.png} &
        \includegraphics[width=0.25\textwidth]{img/seasons/spring/sample_000011_true_rgb.png} \\
        
        \includegraphics[width=0.25\textwidth]{img/seasons/spring/sample_000008_sar_pseudo.png} &
        \includegraphics[width=0.25\textwidth]{img/seasons/spring/sample_000008_pred_rgb.png} &
        \includegraphics[width=0.25\textwidth]{img/seasons/spring/sample_000008_true_rgb.png} \\
        
        \includegraphics[width=0.25\textwidth]{img/seasons/spring/sample_000091_sar_pseudo.png} &
        \includegraphics[width=0.25\textwidth]{img/seasons/spring/sample_000091_pred_rgb.png} &
        \includegraphics[width=0.25\textwidth]{img/seasons/spring/sample_000091_true_rgb.png} \\
    \end{tabular}
    \caption[Qualitative results on the spring subset]{%
    Representative SAR-to-optical translation result on the \textbf{spring} subset. 
    Columns: \textbf{(i)} SAR input (pseudo-RGB), 
    \textbf{(ii)} model-generated optical image, 
    \textbf{(iii)} ground-truth Sentinel-2 image.
    }
    \label{fig:appendix_spring}
\end{figure}


% ---------------- SUMMER ----------------
% \section{Summer Subset}
\newpage

\vspace{2em} 

\begin{table}[h!]
\centering
\caption[Quantitative results on the summer subset]{Quantitative performance of the model on the summer subset of the SEN12-MS dataset.}
\begin{tabular}{lcccc}
\toprule
\textbf{Season} & \textbf{SSIM} & \textbf{PSNR (dB)} & \textbf{LPIPS} & \textbf{SAM (°)} \\
\midrule
Summer & 0.800 & 22.79 & 0.268 & 12.43 \\
\bottomrule
\end{tabular}
\label{tab:summer_results}
\end{table}

\vspace{2em} 

\begin{figure}[h!]
    \centering
    \setlength{\tabcolsep}{2pt}
    \renewcommand{\arraystretch}{1.0}
    \begin{tabular}{c *{3}{c}}

        \includegraphics[width=0.25\textwidth]{img/seasons/summer/sample_000004_sar_pseudo.png} &
        \includegraphics[width=0.25\textwidth]{img/seasons/summer/sample_000004_pred_rgb.png} &
        \includegraphics[width=0.25\textwidth]{img/seasons/summer/sample_000004_true_rgb.png} \\
        
        \includegraphics[width=0.25\textwidth]{img/seasons/summer/sample_000017_sar_pseudo.png} &
        \includegraphics[width=0.25\textwidth]{img/seasons/summer/sample_000017_pred_rgb.png} &
        \includegraphics[width=0.25\textwidth]{img/seasons/summer/sample_000017_true_rgb.png} \\
        
        \includegraphics[width=0.25\textwidth]{img/seasons/summer/sample_000026_sar_pseudo.png} &
        \includegraphics[width=0.25\textwidth]{img/seasons/summer/sample_000026_pred_rgb.png} &
        \includegraphics[width=0.25\textwidth]{img/seasons/summer/sample_000026_true_rgb.png} \\
    \end{tabular}
    \caption[Qualitative results on the summer subset]{%
    Representative SAR-to-optical translation result on the \textbf{summer} subset. 
    Columns: \textbf{(i)} SAR input (pseudo-RGB), 
    \textbf{(ii)} model-generated optical image, 
    \textbf{(iii)} ground-truth Sentinel-2 image.
    }
    \label{fig:appendix_summer}
\end{figure}

\chapter{Value Distributions of Individual Optical Bands}
\label{appendix:data_ranges}
% ----------------------------------------------------------
\begin{figure}[!htbp]
  \centering

  % First histogram
  \adjustbox{max width=0.98\textwidth, max totalheight=0.45\textheight, keepaspectratio}{
    \includegraphics{img/data_ranges/sample_1.png}
  }
  \vspace{0.8em}

  % Second histogram
  \adjustbox{max width=0.98\textwidth, max totalheight=0.45\textheight, keepaspectratio}{
    \includegraphics{img/data_ranges/sample_2.png}
  }
  \label{fig:data_ranges_12}
\end{figure}

\FloatBarrier % ensure next figure pair appears after this one

\begin{figure}[!htbp]
  \centering
  % Third histogram
  \adjustbox{max width=0.95\textwidth, max totalheight=0.45\textheight, keepaspectratio}{
    \includegraphics{img/data_ranges/sample_3.png}
  }
  \vspace{0.8em}

  % Fourth histogram
  \adjustbox{max width=0.95\textwidth, max totalheight=0.45\textheight, keepaspectratio}{
    \includegraphics{img/data_ranges/sample_4.png}
  }

  \caption[Value distributions of individual optical bands]{Value distributions of the individual optical bands.
  The plots show per-band reflectance range and relative scaling across spectral channels.}
  \label{fig:data_ranges_34}
\end{figure}

\FloatBarrier
